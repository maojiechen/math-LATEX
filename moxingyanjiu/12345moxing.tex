\section{12345模型}
\subsection{【模型解读】}

初中几何,直角三角形具有举足轻重的地位,贯彻初中数学的始终,无论是一次函数、平行四边形、特殊平行四边形、反比例函数、二次函数、相似、圆,都离不开直角三角形。而在直角三角形中,345的三角形比含有$30^\circ$的直角三角形的$1:\sqrt{3}:2$以及含有$45^\circ$的直角三角形的$1:1:\sqrt{2}$更加特殊更加重要。因为345不仅仅是自己特殊,更是可以在变化中隐藏更加特殊的变化$(1:2:\sqrt{5}\text{及}1:3:\sqrt{10})$,综合性非常大,深受压轴题的喜爱。现在带领大家领略一下345的独特魅力:

\[ \text{12345模型}\begin{cases}
\tan \alpha=\frac{1}{2}\\

\tan \beta=\frac{1}{3}
\end{cases} \Rightarrow \alpha+\beta=45^\circ \]