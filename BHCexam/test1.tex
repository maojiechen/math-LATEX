\documentclass[marginline,answers]{BHCexam}


\begin{document}
\biaoti{2023年安徽初中学业水平考试}
\fubiaoti{数学试卷}
\maketitle
\notice

\begin{questions}
\xuanze

\question 已知集合~$A=\{x\mid \abs{x-1}<3 \}$,
集合~$B=\{y| y=x^2+2x+1,x\in\mathbb{R}\}$, 则~$A\cap
\complement_U B$~为\xx{D}.

\twoch{$[\,0,4)$}{$(-\infty,-2\,]\cup[4,+\infty)$}{$(-2,0)$}{$(0,4)$}

\question 若~$a$、$b$~是直线, $\alpha$、$\beta$~是平面,
则以下命题中真命题是\stk{D}.\\
\fourch{若~$a$、$b$~异面, $a\subset\alpha$,$b\subset\beta$, 且~$a\perp b$, 则~$\alpha\perp\beta$}{若~$a\parallel b$, $a\subset\alpha$, $b\subset\beta$,则~$\alpha\parallel\beta$}{若~$a\parallel \alpha$,
$b\subset\beta$, 则~$a$、$b$ 异面}{若~$a\perp b$, $a\perp\alpha$,$b\perp\beta$, 则~$\alpha\perp\beta$}


\question 已知集合~$A=\{x\mid \abs{x-1}<3 \}$,
集合~$B=\{y| y=x^2+2x+1,x\in\mathbb{R}\}$, 则~$A\cap
\complement_U B$~为\stk{C}.

\twoch{$[\,0,4)$}{$(-\infty,-2\,]\cup[4,+\infty)$}{$(-2,0)$}{$(0,4)$}

\question 若~$a$、$b$~是直线, $\alpha$、$\beta$~是平面,
则以下命题中真命题是\stk{D}.\\
\fourch{若~$a$、$b$~异面, $a\subset\alpha$,$b\subset\beta$, 且~$a\perp b$, 则~$\alpha\perp\beta$}{若~$a\parallel b$, $a\subset\alpha$, $b\subset\beta$,则~$\alpha\parallel\beta$}{若~$a\parallel \alpha$,
$b\subset\beta$, 则~$a$、$b$ 异面}{若~$a\perp b$, $a\perp\alpha$,$b\perp\beta$, 则~$\alpha\perp\beta$}

\question 已知集合~$A=\{x\mid \abs{x-1}<3 \}$,
集合~$B=\{y| y=x^2+2x+1,x\in\mathbb{R}\}$, 则~$A\cap
\complement_U B$~为\stk{C}.

\twoch{$[\,0,4)$}{$(-\infty,-2\,]\cup[4,+\infty)$}{$(-2,0)$}{$(0,4)$}

\question 若~$a$、$b$~是直线, $\alpha$、$\beta$~是平面,
则以下命题中真命题是\stk{D}.\\
\fourch{若~$a$、$b$~异面, $a\subset\alpha$,$b\subset\beta$, 且~$a\perp b$, 则~$\alpha\perp\beta$}{若~$a\parallel b$, $a\subset\alpha$, $b\subset\beta$,则~$\alpha\parallel\beta$}{若~$a\parallel \alpha$,
	$b\subset\beta$, 则~$a$、$b$ 异面}{若~$a\perp b$, $a\perp\alpha$,$b\perp\beta$, 则~$\alpha\perp\beta$}


\question 已知集合~$A=\{x\mid \abs{x-1}<3 \}$,
集合~$B=\{y| y=x^2+2x+1,x\in\mathbb{R}\}$, 则~$A\cap
\complement_U B$~为\stk{C}.

\twoch{$[\,0,4)$}{$(-\infty,-2\,]\cup[4,+\infty)$}{$(-2,0)$}{$(0,4)$}

\question 若~$a$、$b$~是直线, $\alpha$、$\beta$~是平面,
则以下命题中真命题是\stk{D}.\\
\fourch{若~$a$、$b$~异面, $a\subset\alpha$,$b\subset\beta$, 且~$a\perp b$, 则~$\alpha\perp\beta$}{若~$a\parallel b$, $a\subset\alpha$, $b\subset\beta$,则~$\alpha\parallel\beta$}{若~$a\parallel \alpha$,
	$b\subset\beta$, 则~$a$、$b$ 异面}{若~$a\perp b$, $a\perp\alpha$,$b\perp\beta$, 则~$\alpha\perp\beta$}

\question 已知集合~$A=\{x\mid \abs{x-1}<3 \}$,
集合~$B=\{y| y=x^2+2x+1,x\in\mathbb{R}\}$, 则~$A\cap
\complement_U B$~为\stk{C}.

\twoch{$[\,0,4)$}{$(-\infty,-2\,]\cup[4,+\infty)$}{$(-2,0)$}{$(0,4)$}

\question 若~$a$、$b$~是直线, $\alpha$、$\beta$~是平面,
则以下命题中真命题是\stk{D}.\\
\fourch{若~$a$、$b$~异面, $a\subset\alpha$,$b\subset\beta$, 且~$a\perp b$, 则~$\alpha\perp\beta$}{若~$a\parallel b$, $a\subset\alpha$, $b\subset\beta$,则~$\alpha\parallel\beta$}{若~$a\parallel \alpha$,
	$b\subset\beta$, 则~$a$、$b$ 异面}{若~$a\perp b$, $a\perp\alpha$,$b\perp\beta$, 则~$\alpha\perp\beta$}



\tiankong
\question 已知~$\vec{a}=(k,-9)$、$\bm{b}=(-1,k)$, $\bm{a}$~与~$\bm{b}$~为平行向量,
    则~$k=$\stk{$\pm3$}.

\question 若函数~$f(x)=x^{6m^2-5m-4}\,(m\in\mathbb{Z})$~的图像关于~$y$~轴对称,
    且~$f(2)<f(6)$, 则~$f(x)$~的解析式为\stk{$f(x)=x^{-4}$}.

\question 若~$f(x+1)=x^2\,(x\leq0)$, 则~$f^{-1}(1)=$\stk{0}.

\question 在~$b\text{g}$~糖水中含糖~$a\text{g}$\,($b>a>0$), 若再添加~$m\text{g}$~糖~($m>0$),




\jiandaa
\question 已知复数~$z$ 满足:$\abs{z}-z^*=\dfrac{10}{1-w\textbf{i}}$(其中~$z^*$
是~$z$ 的共轭复数).
\begin{parts}
\part[7] 求复数~$z$;
\part[7] 若复数~$w=\cos\theta+\textbf{i}\sin\theta\,(\theta\in\mathbb{R})$, 求~$\abs{z-2}$ 的取值范围.
\end{parts}

\begin{solution}
\begin{parts}
\part $z=3+4\textbf{i}$
\part $\abs{z-w}\in[4,6]$
\end{parts}
\end{solution}

\question[14] 函数~$f(x)=4\sin\dfrac{\pi}{12}x\cdot\sin
    \left(\dfrac{\pi}{2}+\dfrac{\pi}{12}x\right),x\in[a,a+1]$,
    其中常数~$a\in[0,5]$, 求函数~$f(x)$ 的最大值~$g(a)$.

\begin{solution}
略
\end{solution}


\jiandab
\question[16] 函数~$f(x)=4\sin\dfrac{\pi}{12}x\cdot\sin
    \left(\dfrac{\pi}{2}+\dfrac{\pi}{12}x\right),x\in[a,a+1]$,
    其中常数~$a\in[0,5]$, 求函数~$f(x)$ 的最大值~$g(a)$.

\begin{solution}
略
\end{solution}


\question 已知复数~$z$ 满足:$\abs{z}-z^*=\dfrac{10}{1-w\textbf{i}}$(其中~$z^*$
是~$z$ 的共轭复数).
\begin{parts}
\part[8] 求复数~$z$;若复数~$w=\cos\theta+\textbf{i}\sin\theta\,(\theta\in\mathbb{R})$, 求~$\abs{z-2}$ 的取值范围.若复数~$w=\cos\theta+\textbf{i}\sin\theta\,(\theta\in\mathbb{R})$, 求~$\abs{z-2}$ 的取值范围.
\part[8] 若复数~$w=\cos\theta+\textbf{i}\sin\theta\,(\theta\in\mathbb{R})$, 求~$\abs{z-2}$ 的取值范围.
\end{parts}

\begin{solution}
\begin{parts}
\part $z=3+4\textbf{i}$
\part $\abs{z-w}\in[4,6]$
\end{parts}
\end{solution}


\jiandac
\question[18] 函数~$f(x)=4\sin\dfrac{\pi}{12}x\cdot\sin
    \left(\dfrac{\pi}{2}+\dfrac{\pi}{12}x\right),x\in[a,a+1]$,
    其中常数~$a\in[0,5]$, 求函数~$f(x)$ 的最大值~$g(a)$.

\begin{solution}
略
\end{solution}

\question[18] 函数~$f(x)=4\sin\dfrac{\pi}{12}x\cdot\sin
\left(\dfrac{\pi}{2}+\dfrac{\pi}{12}x\right),x\in[a,a+1]$,
其中常数~$a\in[0,5]$, 求函数~$f(x)$ 的最大值~$g(a)$.

\jiandad
\question[18] 函数~$f(x)=4\sin\dfrac{\pi}{12}x\cdot\sin
\left(\dfrac{\pi}{2}+\dfrac{\pi}{12}x\right),x\in[a,a+1]$,
其中常数~$a\in[0,5]$, 求函数~$f(x)$ 的最大值~$g(a)$.


\jiandae
\question[18] 函数~$f(x)=4\sin\dfrac{\pi}{12}x\cdot\sin
\left(\dfrac{\pi}{2}+\dfrac{\pi}{12}x\right),x\in[a,a+1]$,
其中常数~$a\in[0,5]$, 求函数~$f(x)$ 的最大值~$g(a)$.
\jiandaf
\question[18] 函数~$f(x)=4\sin\dfrac{\pi}{12}x\cdot\sin
\left(\dfrac{\pi}{2}+\dfrac{\pi}{12}x\right),x\in[a,a+1]$,
其中常数~$a\in[0,5]$, 求函数~$f(x)$ 的最大值~$g(a)$.函数~$f(x)=4\sin\dfrac{\pi}{12}x\cdot\sin
\left(\dfrac{\pi}{2}+\dfrac{\pi}{12}x\right),x\in[a,a+1]$,
其中常数~$a\in[0,5]$, 求函数~$f(x)$ 的最大值~$g(a)$.函数~$f(x)=4\sin\dfrac{\pi}{12}x\cdot\sin
\left(\dfrac{\pi}{2}+\dfrac{\pi}{12}x\right),x\in[a,a+1]$,
其中常数~$a\in[0,5]$, 求函数~$f(x)$ 的最大值~$g(a)$.
\end{questions}

\end{document}
