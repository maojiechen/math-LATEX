\section{其他求导类型}
\subsection{隐函数及其求导法}
由方程$F\left(x,y\right)=0$所确定的$y$与$x$间的函数关系称为隐函数。\par
{\kaishu\color{blue} 隐函数求导法:$F\left(x,y\right)=0$两边对$x$求导($y$是$x$的函数$y=f(x)$)得到一个关于$y'_x$的方程,解出$y'_x$即可。}
\begin{example}
求方程$y^5+2y-x-3x^7=0$所确定的隐函数$y$的导数。
\end{example}
\begin{solution}
方程两边对$x$求导\\
$5y^4y'+2y'-1-21x^6=0$\\
$\therefore y'=\dfrac{1+21x^6}{5y^4+2}$
\end{solution}
\begin{example}
求由方程$xy+e^y=e^x$所确定的隐函数$y$的导数$y'(0)$。
\end{example}
\begin{solution}
方程两边对$x$求导$y+xy'+e^yy'=e^x$\\
$\therefore y'=\dfrac{e^x-y}{e^y+x}$\\
$\text{当}x=0\text{时,由方程解出}y=0,\therefore y'(0)=1$
\end{solution}
\begin{example}
设$\ln \sqrt{x^2+y^2}-\arctan \dfrac{y}{x}=\ln 2$~,求$\dfrac{dy}{dx}$。
\end{example}
\begin{solution}
原方程为$\dfrac{1}{2} \ln \left(x^2+y^2\right)=\arctan \dfrac{y}{x}+\ln 2$\\
$\dfrac{x+yy'}{x^2+y^2}=\dfrac{1}{1+\left(\dfrac{y}{x}\right)^2}\dfrac{xy'-y}{x^2}$\\
等号两边对$x$求导得$x+yy'=xy'-y~,\dfrac{dy}{dx}=\dfrac{x+y}{x-y}$
\end{solution}
\begin{example}
求椭圆$\dfrac{x^2}{16}+\dfrac{y^2}{9}=1$在点$\left(2,\dfrac{3\sqrt{3}}{2}\right)$处的切线方程。
\end{example}
\begin{solution}
$\dfrac{2x}{16}+\dfrac{2yy'}{9}=0~,y'=-\dfrac{9x}{16y}~,y'\big|_{\left(2,\tfrac{3\sqrt{3}}{2}\right)}=\left(-\dfrac{9x}{16y}\right)\big|_{\left(2,\tfrac{3\sqrt{3}}{2}\right)}=-\dfrac{\sqrt{3}}{4}$\\
所以,切线方程为$y-\dfrac{3\sqrt{3}}{2}=-\dfrac{\sqrt{3}}{4}\left(x-2\right)$\\
\end{solution}
{\kaishu\color{blue} 注:方程$F\left(x,y\right)=0$中,变量$x$与$y$的地位是平等的,同样可确定$y$的一个隐函数$x$,所以可求$\dfrac{dx}{dy}$。}
\subsection{对数求导法}
先把函数$y=f(x)$取自然对数化为隐函数然后求导,这种方法叫对数求导法。
\begin{example}
设$y=\sqrt{\dfrac{(x-1)(x-2)}{(x-3)(x-4)}}$~,求$y‘$。
\end{example}
\begin{solution}
$\begin{aligned}
x>4\text{时}\\
\ln y&=\dfrac{1}{2}\ln \dfrac{(x-1)(x-2)}{(x-3)(x-4)}\\
&=\dfrac{1}{2}\left[\ln \left(x-1\right)+\ln \left(x-2\right)-\ln \left(x-3\right)-\ln \left(x-4\right)\right]\\
\dfrac{1}{y}y'&=\dfrac{1}{2}\left[\dfrac{1}{x-1}+\dfrac{1}{x-2}-\dfrac{1}{x-3}-\dfrac{1}{x-4}\right]\\
y'&=\dfrac{1}{2}y\left[\dfrac{1}{x-1}+\dfrac{1}{x-2}-\dfrac{1}{x-3}-\dfrac{1}{x-4}\right]\\
&=\dfrac{1}{2}\sqrt{\dfrac{(x-1)(x-2)}{(x-3)(x-4)}}\left[\dfrac{1}{x-1}+\dfrac{1}{x-2}-\dfrac{1}{x-3}-\dfrac{1}{x-4}\right]
\end{aligned}$
\end{solution}
\begin{example}
设$y=\left[f(x)\right]^{g(x)}$~,其中$f(x),g(x)$均为可导函数,且$f(x)>0$~,求$\dfrac{dy}{dx}$。
\end{example}
\begin{solution}
$\ln y=g(x)\ln \left[f(x)\right]$\\
$\dfrac{1}{y}y'=g'(x)\ln \left[f(x)\right]+\dfrac{g(x)}{f(x)}f'(x)$\\
$\therefore \dfrac{dy}{dx}=\left[f(x)\right]^{g(x)}\left\{g'(x)\ln \left[f(x)\right]+g(x)\dfrac{f'(x)}{f(x)}\right\}$
\end{solution}
{\kaishu\color{blue} 注:幂函数也可以写成复合函数的形式求导$y=\left[f(x)\right]^{g(x)}=e^{g(x)\ln \left[f(x)\right]}$。}
\begin{example}
求函数$y=x^{\sin x},(x>0)$~的导数。
\end{example}
\begin{solution}
法一\quad 取对数$\ln y=\sin x\ln x~,\dfrac{1}{y}y'=\cos x\ln x+\dfrac{\sin x}{x}\\
\therefore y'=x^{\sin x}\left[\cos x\ln x+\dfrac{\sin x}{x}\right]$\\
法二\quad$y=x^{\sin x}=\mathrm{e}^{\sin x\ln x}\\
\therefore y'=\mathrm{e}^{\sin x\ln x}\left[\sin x\ln x\right]'=x^{\sin x}\left[\cos x\ln x+\dfrac{\sin x}{x}\right]$
\end{solution}
\begin{example}
设$y=x^x+x^{e^x}+e^x+x^e$~,求$y'$。
\end{example}
\begin{solution}
$y=e^{x\ln x}+e^{e^x\ln x}+e^x+x^e\\
\therefore y'=e^{x\ln x}\left(\ln x+1\right)+e^{e^x\ln x}\left(e^x\ln x+\dfrac{e^x}{x}\right)+e^x+ex^{e-1}\\
y'=x^x\left(\ln x+1\right)+x^{e^x}\left(e^x\ln x+\dfrac{e^x}{x}\right)+e^x+ex^{e-1}$
\end{solution}
\begin{example}
设由方程$x^y=y^x$~确定$y$是$x$~的函数,求$y'$。
\end{example}
\begin{solution}
方程两边取对数$y\ln x=x\ln y$\\
等号两边对$x$求导$y'\ln x+\dfrac{y}{x}=\ln y+\dfrac{x}{y}y'\\
\therefore y'\left(\ln y-\dfrac{x}{y}\right)=\ln y-\dfrac{y}{x}\\
y'=\dfrac{\ln y-\dfrac{y}{x}}{\ln x-\dfrac{x}{y}}$
\end{solution}
\subsection{双曲函数与反双曲函数的导数}
{\kaishu\color{blue} 公式:\[  \left(sh x\right)'=ch x~\quad\left(ch x\right)'=sh x~\quad\left(th x\right)'=\dfrac{1}{ch ^2 x}\]
\[ \left(arshx\right)'=\dfrac{1}{\sqrt{1+x^2}}\quad\left(archx\right)'=\dfrac{1}{\sqrt{x^2-1}}\quad\left(arthx\right)'=\dfrac{1}{1-x^2}\]
注:分段函数的导数,如\[  \begin{dcases}
\begin{aligned}
&x-1,&x\leq 0\\
&2x,&0<x\leq 1\\
&x^2+1,&1<x\leq 2\\
&\dfrac{1}{2}x+4,&x>2
\end{aligned}
\end{dcases},\text{求}f'(x)~.
\]
\begin{solution}
$f(x)$在$x=0$不连续,所以不可导:\[f'_{-}(1)=2,\quad f'_{+}(1)=2,\quad \therefore f'(1)=2,f'_{-}(2)=4,f'_{+}(2)=\dfrac{1}{2},\]所以$f'(2)$不存在。
\[  f'(x)=\begin{dcases}
\begin{aligned}
&1,&x<0\\
&2,&0<x\leq 1\\
&2x,&1<x<2\\
&\dfrac{1}{2},&x>2
\end{aligned}
\end{dcases}
\]
\end{solution}
}
\subsection{高阶导数}
\subsection{参数方程的导数}


